\documentclass[a4paper]{article}
\usepackage{lipsum}
\usepackage{url}
\usepackage{graphicx}
\usepackage[margin=2cm]{geometry}
\graphicspath{ {images/} }

%Custom Commands
\newcommand{\Pokemon}{Pok\'{e}mon}


\begin{document}

%Title Information
\title{
    G53PEC COURSEWORK 2 ― CASE STUDY ANALYSIS \\
    \large{Student Name: Benjamin Charlton \\
            Student ID:\@ 4262648 \\
            Date of Submission: 15\textsuperscript{th} December 2017\\
            Word Count: \\
            Declaration: I confirm that this coursework submission is all my own work, except where explicitly indicated within the text.}
    \date{}}
\maketitle

\section{Project Title and Description}
The project that is going to be analysed is titled Applying Evolutionary Algorithms to \Pokemon{} Team Building.
The project aims to use evolutionary algorithms to evolve a competitive \Pokemon{} team for a user to be able to use.
The project will create a tool which can optimise a team for competitive Pokémon battling.
The objective of the project is to see if Evolutionary Algorithms can solve this type of problem on a more complex data source such as finding good teams from all potential sets of Pokémon.
By using the AI techniques in evolutionary algorithms the aim is to be able to provide a strong team that will be comparable with the expert designed teams.
This would be extremely helpful for newer players to get into \Pokemon{} who are potentially looking to compete at a high level such as the World Championships\cite{worldsOverview}.
\par
This project is my 3rd year dissertation project that I am currently developing.
In its current state as a dissertation project, there are few ethical issues surrounding it.
The ethical issues that will be discussed come from looking at the project as if it was extended into a commercial release.

\section{Ethical Issues}
\subsection{Privacy of Users}
The majority of modern software collect some user data in some form or another, this typically happens via a login system.
For this example it is assumed that the final product would contain some form of login system which then would be used to store users settings and saved data from the system on the cloud, as well as allowing the company to contact users via an email newsletter.
This would be beneficial to the users by allowing easy access from different devices if required and to be kept up to date on the latest features.
\\ \par
Solove discussed the issues with privacy and classified the potential issues into a taxonomy.
In particular the section B discusses information processing, which `refers to the use, storage, and manipulation of data that has been collected'\cite{soloveTaxonomy}.
This is relevant as it discusses ways that the data is used by the company, which have clear steps to mitigate these issues.
The areas that will be looked into particular are:
\begin{itemize}
    \item Aggregation - Linking data together giving a better idea of a person.
    \item Identification - Linking data with a identifiable individual.
    \item Insecurity - Improper security resulting in private data being accessed by unwanted people.
    \item Secondary Use - Using data for reasons it wasn't first collected for.
    \item Exclusion - Not notifying users that data has been collected or allowing for them to correct it.
\end{itemize}
Several of these issues also compound upon each other, for example if data is identifiable and is then aggregated the impact of this is that more of the user data is then identifiable.
\par
In this case study aggregation would come from the linking of data of the users together, such as settings, saved data, login details, and logs.
Individually these items might not be that interesting on their own, but combined they could be used to create a more detailed impression of the individual.
Solove states that this is not desirable as `It reveals facts about data subjects in ways far beyond anything they expected when they gave out the data'\cite{soloveTaxonomy}.
\par
The issue of identification would arise if a enough unique information about a user is collected that it then becomes identifiable.
This data on its own isn't helpful or beneficial to anyone but as it starts combining with other data it can cause issues.
\par
Insecurity allows for unwanted parties to be able to access the private data.
A user may be comfortable with the developers having access to the data to help run the project but that trust and comfort doesn't extend to everybody particularly unknown parties.
\par
Secondary use could go against the users permission to what the data was originally intended for.
This would break the trust between the user and the developers and is morally wrong as it would mean the developers are breaking their promise of what the data will be used for.
\par
Exclusion to many people is the worst of the issues in this taxonomy.
This is due to the deceitful nature of taking data from users with out their knowledge.
The inability to correct data is also troubling as it would lead to the user being misrepresented.
\\ \par
Developers on the project have an ethical obligation to consider the users privacy, this can be seen in the BCS code of conduct (1a.)\cite{bscCoC}.
The code of conduct is rather vague with what actions the developers should take but it does make it clear that they have an ethical obligation which has been agreed upon by the community.
\par
They also have a legal obligation, specifically due to the Data Protection Act in the UK\cite{dataProtectionAct}.
This act details what data can be stored, how long for, what permissions are required and much more.
The developers would be ethically obligated to follow this by the BCS code of conduct (2d.)\cite{bscCoC}.

\subsection{Intellectual Property Infringement}
A large concern of this project is the intellectual property infringement as the project is heavily based upon the \Pokemon{} series.
As the fundamental aims and objectives are so heavily linked to the franchise it will be impossible to create the project without it.
This does not mean that the project could not become a commercial product without intellectual property issues but the creators would have to be extremely careful about what assets they use and how it is marketed.
\par
The UK government say that `Intellectual property is something unique that you physically create. An idea alone is not intellectual property'\cite{ukGovIP}.
This is to say that you can't claim you own an intellectual property until it has manifested in some form.
\\ \par
It could be argued that the developers have a ethical duty to not cause intellectual property infringement beyond the legal duty they have.
In the BCS code of conduct it states that members should `have due regard for the legitimate rights of Third Parties'\cite{bscCoC} where in this example the legitimate rights are those pertaining to any intellectual property.
Either way you see the situation in the eyes of rule deontology their is some need for the company to make sure they are careful to not infringe on the intellectual property.
\\ \par
There are several points in the project that the could be considered to be intellectual property.
The following parts will be considered:
\begin{itemize}
    \item Art and Sound assets - Art and sounds from the video game series and promotional material, originally created by The \Pokemon{} Company.
    \item Names of Characters - Using names of characters from the series to identify them.
\end{itemize}
\par
These various parts encompass several parts of intellectual property particularly copyright and trademark.

\section{Proposed Way Forward}
\subsection{Privacy of Users}
To combat the ethical issues brought forward by any privacy concerns

\subsection{Intellectual Property Infringement}
One simple way forward with this is to contact the holder of the intellectual property (The \Pokemon{} Company in this example) to ask for permission and use of the property in question.
This often happens with some larger tools and often the property is shared in agreement with a contract, either limiting the use or some financial exchange.
In this example it would be impossible to acquire such permission as The \Pokemon{} Company states in its terms of use the following `Because we receive thousands of such requests, our policy is to decline use of our trademarks and copyrights'\cite{pokemonTOS}.
It could be argued that this is not required due to fair use, the rest of the proposed ways go forward under the notion of fair use and what is allowed due to fair use.
\\ \par
If the project was released with assets taken directly from The \Pokemon{} Company to go alongside the final data to help display it nicely, it would be unethical as it was breaking the copyright law.
This would not be covered by fair use as the assets would not be used in a transformative way.
The additions of the copyrighted assets would grant the user some happiness so it makes sense to look at this under utilitarianism but when you look at rule utilitarianism it states that this happiness comes from actions that apply with the rules, which this would not.
It would also be difficult to argue that the project could create its own assets to use as the characters themselves would be protected under copyright.
This has been shown to be enforced by The \Pokemon{} Company as they have previously removed apps from mobile phone app stores due to copyright infringement\cite{pokedexCopyright}.
\par
When regarding trademarks the situation becomes more debatable.
The International Trademark Association defines two types of fair use with trademarks, descriptive and nominative.
This tackles nominative fair use as the trademarked terms aren't descriptive factors.
The International Trademark Association states that `Nominative fair use generally is permissible as long as the product or service in question is not readily identifiable without use of the trademark' and that the `use of the mark does not suggest sponsorship or endorsement by the trademark owner'\cite{ITA}.
This states that it would be fair use to use the names of \Pokemon{} as it would be infeasible to identify them without using the trademarked names.
To further compile and make sure that things are clear it would be proposed that the project is transparent and says that the trademarks belong to The \Pokemon{} Company as to say that the product is not endorsed.
This is morally and ethically backed up by rule deontology as the proposed actions are in compliance with any rules and have been guided by these rules.

\bibliography{G53PEC}
\bibliographystyle{plain}

\end{document}
