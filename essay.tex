\documentclass[a4paper]{article}
\usepackage{lipsum}
\usepackage{url}
\usepackage{graphicx}
\usepackage[margin=2cm]{geometry}
\graphicspath{ {images/} }

%Custom Commands
\newcommand{\Pokemon}{Pok\'{e}mon}


\begin{document}

%Title Information
\title{
    G53PEC COURSEWORK 2 ― CASE STUDY ANALYSIS \\
    \large{Student Name: Benjamin Charlton \\
            Student ID:\@ 4262648 \\
            Date of Submission: 15\textsuperscript{th} December 2017\\
            Word Count: \\
            Declaration: I confirm that this coursework submission is all my own work, except where explicitly indicated within the text.}
    \date{}}
\maketitle

\section{Project Title and Description}
The project that is going to be analysed is titled Applying Evolutionary Algorithms to \Pokemon{} Team Building.
The project aims to use evolutionary algorithms to evolve a competitive \Pokemon{} team for a user to be able to use.
The project will create a tool which can optimise a team for competitive Pokémon battling.
The objective of the project is to see if Evolutionary Algorithms can solve this type of problem on a more complex data source such as finding good teams from all potential sets of Pokémon.
By using the AI techniques in evolutionary algorithms the aim is to be able to provide a strong team that will be comparable with the expert designed teams.
This would be extremely helpful for newer players to get into \Pokemon{} who are potentially looking to compete at a high level such as the World Championships\cite{worldsOverview}.
\par
This project is my 3rd year dissertation project that I am currently developing.
In its current state as a dissertation project, there are few ethical issues surrounding it.
The ethical issues that will be discussed come from looking at the project as if it was extended into a commercial release.

\section{Ethical Issues}
\subsection{Privacy of Users}
The majority of modern software collect some user data in some form or another, this typically happens via a login system.
For this example it is assumed that the final product would contain some form of login system which then would be used to store users settings and saved data from the system on the cloud, as well as allowing the company to contact users via an email newsletter.
This would be beneficial to the users by allowing easy access from different devices if required and to be kept up to date on the latest features.
\\ \par
Solove discussed the issues with privacy and classified the potential issues into a taxonomy.
In particular the section B discusses information processing, which `refers to the use, storage, and manipulation of data that has been collected'\cite{soloveTaxonomy}.
This is relevant as it discusses ways that the data is used by the company, which have clear steps to mitigate these issues.
The areas that will be looked into particular are:
\begin{itemize}
    \item Aggregation - Linking data together giving a better idea of a person.
    \item Identification - Linking data with a identifiable individual.
    \item Insecurity - Improper security resulting in private data being accessed by unwanted people.
    \item Secondary Use - Using data for reasons it wasn't first collected for.
    \item Exclusion - Not notifying users that data has been collected or allowing for them to correct it.
\end{itemize}
Several of these issues also compound upon each other, for example if data is identifiable and is then aggregated the impact of this is that more of the user data is then identifiable.
\par
In this case study aggregation would come from the linking of data of the users together, such as settings, saved data, login details, and logs.
Individually these items might not be that interesting on their own, but combined they could be used to create a more detailed impression of the individual.
Solove states that this is not desirable as `It reveals facts about data subjects in ways far beyond anything they expected when they gave out the data'\cite{soloveTaxonomy}.
\par
The issue of identification would arise if a enough unique information about a user is collected that it then becomes identifiable.
This data on its own isn't helpful or beneficial to anyone but as it starts combining with other parts of the
\\ \par
Developers on the project have an ethical obligation to consider the users privacy, this can be seen in the BCS code of conduct (1a.)\cite{bscCoC}.
The code of conduct is rather vague with what actions the developers should take but it does make it clear that they have an ethical obligation which has been agreed upon by the community.
\\ \par
DATA PROTECTION, LEGAL ACTS


\section{Proposed Way Forward}

\bibliography{G53PEC}
\bibliographystyle{plain}





\end{document}
